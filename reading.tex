\documentclass{article}

\usepackage[notes]{optional}
\usepackage[left=0.8in,top=0.9in,right=0.8in,bottom=1in,nohead]{geometry}
\usepackage{url,hyperref,comment,amsmath}
\usepackage[hyphenbreaks]{breakurl}

%% this gives absolute numbering of enumerated items %%
\newcounter{myEnumCounter}
\newcounter{mySaveCounter}
\renewenvironment{enumerate}{%
  \begin{list}{[\arabic{myEnumCounter}]}{\usecounter{myEnumCounter}%
  \setcounter{myEnumCounter}{\value{mySaveCounter}}}
  }{%
  \setcounter{mySaveCounter}{\value{myEnumCounter}}\end{list}%
}
\newcommand\myEnumReset{\setcounter{mySaveCounter}{0}}

\renewcommand{\labelitemi}{$\triangleright$}
\renewcommand{\labelitemii}{$\diamond$}
\renewcommand{\labelitemiv}{$\cdot$}

\opt{notes}{\newenvironment{notes}{\vspace{-0.4em}\begin{itemize}\small}{\end{itemize}}}
\opt{-notes}{\excludecomment{notes}}

\newcommand{\URL}[1]{{\small\burl{#1}}}

\title{
{\Large\it Major Area Exam}\\
Reading List
}
\author{
Stefan Karpinski\\
Department of Computer Science\\
University of California, Santa Barbara\\
\texttt{sgk@cs.ucsb.edu}
}

\begin{document}
\maketitle

\section{Reference}

\begin{enumerate}

\bibitem{Stevens93}
W.~Stevens.
\textit{TCP/IP Illustrated, Volume 1: The Protocols}.
Addison-Wesley Professional, 1st edition, 1993.

\bibitem{Tanenbaum96}
A.~Tanenbaum. \textit{Computer Networks}. Prentice Hall, Upper Saddle River NJ, 3rd edition, 1996.

\end{enumerate}

\section{General Networking}

\begin{enumerate}

\bibitem{Clark88}
D.~Clark. The Design Philosophy of the DARPA Internet Protocols. \textit{Proceedings of ACM SIGCOMM '88}, 1988.

\bibitem{Saltzer84} J.~Saltzer, D.~Reed,  D.~Clark. End-to-End Arguments in System Design. \textit{ACM Transactions on Computer Systems}, 1984.

%\bibitem{Paxson97:end-to-end}
%V.~Paxson.
%End-to-end routing behavior in the Internet.
%\textit{IEEE/ACM Transactions on Networking}, October 1997.

\bibitem{Paxson97}
V.~Paxson, S.~Floyd.
Why we don't know how to simulate the Internet.
\textit{Proceedings of 1997 Winter Simulation Conference}, Atlanta, GA, December 1997.

\bibitem{Faloutsos99}
M.~Faloutsos, P.~Faloutsos, C.~Faloutsos.
On power-law relationships of the Internet topology.
\textit{Proceedings of ACM SIGCOMM'99}, Cambridge MA,
  August 1999.

\bibitem{Medina00}
A.~Medina, I.~Matta, J.~Byers.
On the origin of power laws in Internet topologies.
\textit{ACM Computer Communication Review}, April 2000.

\bibitem{Zegura97}
E.~Zegura, K.~Calvert, M.~Donahoo.
A quantitative comparison of graph-based models for Internet topology.
\textit{IEEE/ACM Transactions on Networking}, December 1997.

\bibitem{Doar97}
M.~Doar.
A better model for generating test networks.
\textit{Proceedings of IEEE Global Internet (GLOBECOM'97)}, London, UK, November 1997.
  
\bibitem{Satyanarayanan96}
M.~Satyanarayanan.
Fundamental challenges in mobile computing.
\textit{Proceedings of ACM Symposium on Principles of Distributed Computing}, Philadelphia PA, May 1996.

\bibitem{Macker98}
J.~Macker, M.~Corson.
Mobile ad hoc networking and the IETF.
\textit{ACM Mobile Computing and Communications Review}, 1998.

\bibitem{Perrone03}
L.~Perrone, Y.~Yuan, D.~Nicol.
Modeling and Simulation Best Practices for Wireless Ad hoc Networks.
\textit{Proceedings of the Winter Simulation Conference}, December 2003.

\end{enumerate}

\section{Mathematical Foundations}

\begin{enumerate}

\bibitem{Ash65}
R.~Ash. \textit{Information Theory}. Interscience Publishers, New York NY, 1965.

\bibitem{Harvey93}
A.~Harvey.
\textit{Time Series Models}.
The MIT Press, Cambridge MA, 2nd edition, 1993.

\bibitem{Dubois93}
D.~Dubois, H.~Prade.
Fuzzy sets and probability: Misunderstandings, bridges and gaps.
\textit{Proceedings of Second IEEE International Conference on Fuzzy Systems}, Piscataway NJ, 1993.

\bibitem{Box94}
G.~Box, G.~Jenkins.
\textit{Time Series Analysis: Forecasting \& Control}.
Holden-Day, 2nd edition, 1994.

\bibitem{Kokoszka95}
P.~Kokoszka, M.~Taqqu.
Fractional ARIMA with Stable Innovations.
\textit{Stochastic Processes and their Applications}, 1995.

\bibitem{Adler98}
R.~Adler, R.~Feldman, M.~Taqqu.
\textit{A Practical Guide to Heavy Tails: Statistical Techniques and Applications}.
Birkhauser, Boston MA, 1998.

\bibitem{Carlson99}
J.~Carlson, J.~Doyle. Highly optimized tolerance: A mechanism for power laws in designed systems.
\textit{Physical Review E}, August 1999.

\bibitem{Ross00}
S.~Ross.
\textit{Introduction to probability and statistics for engineers and scientists}.
Harcourt Academic Press, San Diego CA, 2000.

\bibitem{Downey01}
A.~Downey.
The structural cause of file size distributions.
\textit{Proceedings of the Ninth International Symposium in Modeling, Analysis and Simulation of Computer and Telecommunication Systems (MASCOTS’01)}, 2001.

\bibitem{Ross02}
S.~Ross.
\textit{Probability models for computer science}.
Harcourt Academic Press, San Diego CA, 2002.

\end{enumerate}

\section{Node Behavior: Analysis, Modeling and Prediction}

\subsection{Wired Networks}

\begin{enumerate}

\bibitem{Leland93}
W.~Leland, M.~Taqq, W.~Willinger, D.~Wilson.
On the self-similar nature of Ethernet traffic.
\textit{Proceedings of ACM SIGCOMM'93}, San Francisco CA, 1993.

\bibitem{Paxson95}
V.~Paxson, S.~Floyd.
Wide area traffic: the failure of poisson modeling.
\textit{IEEE/ACM Transaction on Networking}, 1995.

\bibitem{Taqqu97}
S.~Taqqu, W.~Willinger, R.~Sherman.
Proof of a fundamental result in self-similar traffic modeling.
\textit{ACM SIGCOMM Computer Communication Review}, 1997

\bibitem{Willinger98}
W.~Willinger, V.~Paxson.
Where mathematics meets the Internet.
\textit{Notices of the American Mathematical Society}, August 1998.

\bibitem{Wolski98}
R.~Wolski.
Dynamically forecasting network performance using the network weather service.
\textit{Journal of Cluster Computing}, 1998.

\bibitem{Fischer01}
M.~Fischer, T.~Fowler.
Fractals, Heavy-Tails and the Internet.
\textit{Sigma Technology Summaries, 2 (Summer 2001)}, Mitretek Systems Technical Report, 2001.

\bibitem{Brownlee02}
N.~Brownlee, K.~Claffy.
Understanding Internet traffic streams: Dragonflies and tortoises.
\textit{IEEE Communications Magazine}, October 2002.

\bibitem{Saroiu02}
S.~Saroiu, P.~Gummadi, S.~Gribble.
A Measurement Study of Peer-to-Peer File Sharing Systems.
Technical Report UW-CSE-01-06-02, Department of Computer Science \& Engineering, University of Washington, 2002.

\bibitem{Willinger02}
W.~Willinger, R.~Govindan, S.~Jamin, V.~Paxson, S.~Shenker.
Scaling phenomena in the Internet: Critically examining criticality.
\textit{Proceedings of the National Academy of Sciences of the United States of America}, February 2002.

\bibitem{Zhang02}
Y.~Zhang, L.~Breslau, V.~Paxson, S.~Shenker.
On the Characteristics and Origins of Internet Flow Rates.
\textit{Proceedings of the 2002 conference on Applications, technologies, architectures, and protocols for computer communications}, 2002.

\bibitem{Estan03}
C.~Estan, S.~Savage, G.~Varghese.
Automatically Inferring Patterns of Resource Consumption in Network Traffic.
\textit{Proceedings of SIGCOMM}, Karlsruhe, Germany, 2003.

\bibitem{Brevik04}
J.~Brevik, D.~Nurmi, R.~Wolski.
Automatic methods for predicting machine availability in desktop grid and peer-to-peer systems.
\textit{Proceedings of Fourth International Scientific Workshop on Global and Peer-to-Peer Computing}, Chicago IL, 2004.

\bibitem{Karagiannis04}
T.~Karagiannis, M.~Molle, M.~Faloustos, A.~Broido.
A nonstationary poisson view of Internet traffic.
\textit{Proceedings of IEEE INFOCOM 2004}, March 2004.

\bibitem{Uhlig04}
S.~Uhlig.
Non-stationarity and high-order scaling in TCP flow arrivals: a methodological analysis.
\textit{ACM SIGCOMM Computer Communications Review}, April 2004.

\bibitem{Sen04}
S.~Sen, J.~Wang.
Analyzing Peer-To-Peer Traffic Across Large Networks.
\textit{IEEE/ACM Transactions on Networking}, April 2004.

\bibitem{Wright04}
C.~Wright, F.~Monrose, G.~Masson.
HMM Profiles for Network Traffic Classification (Extended Abstract).
\textit{Proceedings of VizSEC/DMSEC’04}, Washington DC, October 2004.

\bibitem{Yin05}
H.~Yin, C.~Lin, B.~Sebastien, B.~Li, G.~Min.
Network traffic prediction based on a new time series model.
\textit{International Journal of Communication Systems}, February 2005.

\bibitem{Karagiannis05}
T.~Karagiannis, K.~Papagiannaki, M.~Faloutsos.
BLINC: Multilevel Traffic Classification in the Dark.
\textit{Proceedings of ACM SIGCOMM}, August 2005.

\bibitem{Brownlee05}
N.~Brownlee.
Some Observations of Internet Stream Lifetimes.
\textit{Passive and Active Measurement Workshop}, 2005

\bibitem{Harmantzis05}
F.~Harmantzis, D.~Hatzinakos.
Heavy Network Traffic Modeling and Simulation using Stable FARIMA Processes.
Not yet published (see \URL{http://personal.stevens.edu/~fharmant/Publish/Stochastics/HHitc05.PDF}).

\bibitem{Gong05}
W.~Gong, Y.~Liu, V.~Misra.
Self-Similarity and Long Range Dependence on the Internet: A Second Look at the Evidence, Origins and Implications.
Not yet published (see \URL{http://www-net.cs.umass.edu/~yongliu/papers/cn.pdf}).

\bibitem{Shi05}
W.~Shi, M.~MacGregor, P.~Gburzynski.
Synthetic Trace Generation for the Internet: An Integrated Model.
Not yet published (see \URL{http://www.cs.ualberta.ca/~pawel/PAPERS/spects04.pdf}).

\bibitem{Lan05}
K.~Lan, J.~Heidemann.
On the correlation of Internet flow characteristics.
Not yet published (see \URL{http://www.cse.unsw.edu.au/~klan/paper/flow.pdf}).

\end{enumerate}

\subsection{Wireless Networks}

\begin{enumerate}

\bibitem{Liu98}
T.~Liu, P.~Bahl, I.~Chlamtac.
Mobility modeling, location tracking, and trajectory prediction in wireless ATM networks.
\textit{IEEE Journal on Selected Areas in Communications}, 1998.

\begin{notes}
\item Early cell network research on predicting user mobility.
\item Useful for predicting the next cell more accurately, but doesn't provide an overall picture of node mobility.
\item Uses Kalman filter to predict the next cell station.
\end{notes}

\bibitem{Tang99}
D.~Tang, M.~Baker. Analysis of a Metropolitan-Area Wireless Network.
\textit{Proceedings of Mobicom 1999}, Seattle WA, Aug 1999.

\begin{notes}
\item Analysis of seven-week trace of the Metricom metropolitan-area packet radio wireless network.
\item Examines the following features:
\begin{itemize}
	\item number of active nodes over days and hours
	\item frequency and distance of node movement
\end{itemize}
\item Uses clustering to classify radio stations by proximity, and mobile nodes by:
\begin{itemize}
	\item total number of locations
	\item average number of location changes per active day
	\item total number of active days
	\item average number of active days per active week
	\item average number of queries per day
\end{itemize}
\item No resulting models, just descriptive analysis
\end{notes}

\bibitem{Tang00}
D.~Tang, M.~Baker. Analysis of a local-area wireless Network. \textit{Proceedings of Mobicom 2000}, Boston MA, 2000.

\begin{notes}
\item Analysis of a 12-week trace of a WLAN at the Gates Computer Science Building at Stanford.
\item Examines the following aggregate network features:
\begin{itemize}
	\item active users per time of day
	\item active users per day of week
	\item number of active days per user
	\item number of APs visited per user
	\item number of users per AP
	\item number of handoffs per AP
	\item throughput by bytes, packets, APs, applications, and users
\end{itemize}
\item Examines the following ``user traffic characteristics'':
\begin{itemize}
	\item application breakdown by users, packets/bytes, in/out
	\item classes of application mixture profiles (by eye, not clustering)
\end{itemize}
\item Purely descriptive, no models.
\end{notes}

\bibitem{Kotz02}
D.~Kotz, K.~Essien.
Analysis of a campus-wide wireless network.
\textit{Proceedings of ACM MobiCom 2002}, September 2002.

\begin{notes}
\item Analysis of traces from WLAN at Dartmouth in the Fall term of 2001.
\item Based on syslog, snmp, and sniffed traffic ($\text{node}=\text{card}$).
\item Traffic:
\begin{itemize}
	\item total traffic volume
	\item amount of traffic per card
	\item variance across hours, days, weekdays
\end{itemize}
\item Users and Mobility:
\begin{itemize}
	\item number of cards
	\item number of active days per card
	\item number of APs per card
	\item number of buildings per card
\end{itemize}
\item Card (Node) Activity:
\begin{itemize}
	\item number of active cards
	\item times cards are active
	\item session duration
	\item number of sessions per day
	\item distribution of sessions per building
	\item roaming: portion of sessions, how often per session
\end{itemize}
\item AP Activity:
\begin{itemize}
	\item number of APs
	\item times APs are active
	\item distribution of activity across APs
\end{itemize}
\item Buildings...
\item Protocols...
\item Basically, it's a laundry list of descriptive questions.
\end{notes}

%\bibitem{Balachandran02:hot-spots}
%A.~Balachandran, P.~Bahl, G.~Voelker.
%Hot-Spot Congestion Relief in Public-area Wireless Networks.
%\textit{Fourth IEEE Workshop on Mobile Computing Systems and Applications}, Callicoon NY, June 2002.

\bibitem{Balachandran02}
A.~Balachandran, G.~Voelker, P.~Bahl, V.~Rangan.
Characterizing User Behavior and Network Performance in a Public Wireless LAN.
\textit{Proceedings of ACM SIGMETRICS'02}, Marina Del Ray CA, June 2002.

\begin{notes}
\item Analysis of public WLAN trace recorded over 3 days at SIGCOMM at UCSD in August 2001.
\item Aims to ``characterize user behavior and network performance in a public WLAN environment.''
\item User behavior:
\begin{itemize}
	\item connection session length
	\item user distribution across APs
	\item mobility patterns
	\item application mixture
	\item bandwidth requirements
\end{itemize}
\item Network performance:
\begin{itemize}
	\item overall, individual AP load
	\item packet errors, retransmissions
\end{itemize}
\item Major claims/findings:
\begin{itemize}
	\item mobility follows a two-state Markov-Modulated Poisson Process (MMPP).
	\item session length is Pareto-distributed.
	\item load distribution across APs is uneven.
	\item AP workload does not follow number of users.
\end{itemize}
\item Major criticisms:
\begin{itemize}
\item They present \textit{no evidence} for their claim that mobility follows a two-state MMPP model other than stating it repeatedly and referring to previous work. Moreover, the previous work (not available online) is for a model of \textit{aggregate traffic} in mobile networks, not mobility itself. Moreover, it was published in 1996, at which point there was no actual public WLAN data on which to base realistic models.
\item Their claim that session duration is Pareto is also not backed up---they don't even show the PDF or CDF of the distribution or mention any quality of fit measures.
\end{itemize}
\item This paper makes two claims that relate to user behavior modeling, but both claims are unsubstantiated and given without any trace of evidence.
\end{notes}

\bibitem{Camp02}
T.~Camp, J.~Boleng, V.~Davies.
A survey of mobility models for ad hoc network research.
\textit{Wireless Communications and Mobile Computing}, September 2002.

\begin{notes}
\item Individual entity models:
\begin{itemize}
	\item random walk (and derivatives)
	\item random waypoint
	\begin{itemize}
		\item discusses density waves but not speed decay~\cite{Yoon03:speed-decay,Yoon03:sound-models}.
	\end{itemize}
	\item random direction
	\begin{itemize}
		\item travel to edge, bounce off
		\item avoids density waves (and speed decay)
	\end{itemize}
	\item boundless simulation area
	\begin{itemize}
		\item rectangular simulation area topologically transformed in a torus
		\item speed, direction adjusted every unit of time by uniform random delta values
	\end{itemize}
	\item Gauss-Markov mobility model
	\begin{itemize}
		\item speed, direction modeled as ARMA(1) random time-series
		\item nodes are forced away from simulation edges
	\end{itemize}
	\item probabilistic random walk
	\begin{itemize}
		\item velocity: $(x,y)\in\{-1,0,+1\}^2$
		\item $x,y$ modeled as independent three-state Markov processes
		\item the Markov transition matrix is the simulation ``control knob''
	\end{itemize}
	\item city section model
	\begin{itemize}
		\item city modeled as uniform grid
		\item streets have different speed limits
		\item destination chosen at random
		\item path chosen is the fastest
	\end{itemize}
\end{itemize}
\item Group mobility models
\begin{itemize}
	\item exponential correlated model
	\item column model: uniformly moving nodes in some formation
	\item nomadic community model
	\begin{itemize}
		\item group reference point moves according to one model
		\item each node has own space and moves within that
	\end{itemize}
	\item pursue model: a group follows a moving target
	\item reference point model
	\begin{itemize}
		\item group movements centered on a moving center point
		\item not clear on how this differs from nomadic community
		\item uses random waypoint for both levels of mobility
	\end{itemize}
\end{itemize}
\item Simulation results: choice of mobility model impacts protocol performance.
\end{notes}

\bibitem{Yoon03:speed-decay}
J.~Yoon, M.~Liu, B.~Noble.
Random Waypoint Considered Harmful.
\textit{Proceedings of INFOCOM 2003}, San Francisco, April 2003.

\begin{notes}
\item Discovered that node speed decays in random waypoint model.
\item Investigate impact and ways to mitigate the problem.
\end{notes}

\bibitem{Balazinska03}
M.~Balazinska, P.~Castro.
Characterizing mobility and network usage in a corporate wireless local-area network.
\textit{Proceedings of ACM MobiSys 2003}, May 2003.

\begin{notes}
\item Analysis of four week trace gathered on a corporate WLAN (1366 users).
\item Features examined:
\begin{itemize}
	\item distribution of load across APs
	\item descriptive analysis of users
	\item correlation between users and load
	\item correlation between time of day and load
	\item distribution of average individual transfer rates
	\item number of APs visited per user
	\item fraction of time users spend at ``home location''
	\item AP prevalence and persistence wrt users
		$\Rightarrow$ this merits more investigation
\end{itemize}
\item Important results:
\begin{itemize}
	\item average user transfer rate found to follow a clean power law, possibly Pareto
	\item user transfer rates \textit{are} dependent on location, and thus mobility
	\item interesting results on prevalence and persistence of APs wrt users
\end{itemize}
\end{notes}

\bibitem{Yoon03:sound-models}
J.~Yoon, M.~Liu, B.~Noble.
Sound Mobility Models.
\textit{Proceedings of ACM MobiCom'03}, San Diego CA, September 2003.

\begin{notes}
\item Broad class of random mobility models suffer from decay.
\item Ways to avoid the decay phenomenon:
\begin{itemize}
	\item using a warmup period is a partial solution
	\item pick initial parameter values from their marginal distributions
\end{itemize}
\end{notes}

\bibitem{Jardosh03}
A.~Jardosh, E.~Belding-Royer, K.~Almeroth, S.~Suri.
Towards Realistic Mobility Models For Mobile Ad hoc Networks.
\textit{Proceedings of ACM MobiCom'03}, San Diego CA, September 2003.

\begin{notes}
\item Obstacle mobility model:
\begin{itemize}
	\item obstacles are polygonal shapes
	\item paths are given by Vornoi graph for obstacle vertices
	\item intersection of graph with obstacles defines ``doorways''
	\item other graph intersections define destination points (interior, exterior)
\end{itemize}
\item Line-of-sight model for transmission through obstacles.
\end{notes}

%\bibitem{Chowell03}
%G.~Chowell, J.~Hyman, S.~Eubank, C.~Castillo-Chavez.
%Scaling laws for the movement of people between locations in a large city.
%\textit{Physical Review E}, December 2003.

\bibitem{Chinchilla04}
F.~Chinchilla, M.~Lindsey, M.~Papadopouli.
Analysis of wireless information locality and association patterns in a campus.
\textit{Proceedings of IEEE INFOCOM 2004}, March 2004.

\begin{notes}
\item Analysis of traces from UNC campus WLAN, February through April 2003 (7,694 clients).
\item Primary focus on data locality and possible caching paradigms.
\item Secondary focus on user association and mobility patterns.
\item Using Markov chains, 86\% accuracy in predicting next AP association.
\begin{itemize}
	\item individual transition matrix is maintained \textit{for each mobile node}
	\item Markov state determined by either most recent APs those 24 hours prior	\item how much of the remaining 14\% accuracy is due to inherent randomness?
	\item is it even possible to know this? can we determine inherent randomness?
\end{itemize}
\item Also looked at revisit probabilities by client and by AP.
\end{notes}

\bibitem{Schwab04}
D.~Schwab, R.~Bunt.
Characterizing the use of a campus wireless network.
\textit{Proceedings of IEEE INFOCOM 2004}, March 2004.

\begin{notes}
\item Analysis of trace of WLAN at University of Saskatchewan, from one week in January 2003 (134 users).
\item This is another grab-bag of random figures and statistics: no models.
%\item I can't believe this crap got into INFOCOM.
\end{notes}

\bibitem{Zheng04}
Q.~Zheng, X.~Hong, S.~Ray.
Recent Advances in Mobility Modeling for Mobile Ad Hoc Network Research.
\textit{Proceedings of ACMSE’04}, Huntsville AL, April 2004.

\begin{notes}
\item Content-free survey of mobility models.
\item Covers the same models as Camp's survey~\cite{Camp02}, but briefly mentions the obstacle model.
\item Doesn't go into nearly the same depth or detail, except for RWP.
\end{notes}

%\bibitem{Madsen04}
%T.~Madsen, F.~Fitzek, R.~Prasad.
%Impact of Different Mobility Models On Connectivity Probability of a Wireless Ad Hoc Network.
%\textit{Proceedings of International Workshop on Wireless Ad Hoc Networks}, June 2004.

\bibitem{Henderson04}
T.~Henderson, D.~Kotz, I.~Abyzov.
The changing usage of a mature campus-wide wireless network.
\textit{Proceedings of ACM MobiCom 2004}, September 2004.

\begin{notes}
\item Revisit of Dartmouth campus WLAN previously analyzed by Kotz~\cite{Kotz02}.
\item Utilization and mobility increased; most other characteristics remained static.
\end{notes}

\bibitem{Meng04}
X.~Meng, S.~Wong, Y.~Yuan, S.~Lu.
Characterizing flows in large wireless data networks.
\textit{Proceedings of ACM MobiCom 2004}, September 2004.

\begin{notes}
\item Analysis of multiple wireless traces (Dartmouth '02, UCSD '01, IBM '02, Stanford '00).
\item Analysis of flows rather than packets or bytes.
\item Establishes convincing distribution fits for the following:
\begin{itemize}
	\item Interarrival times: Lognormal
	\item Residing times: Lognormal
	\item Number of APs visited: Weibull/Gamma
\end{itemize}
\end{notes}

\bibitem{Musolesi04}
M.~Musolesi, S.~Hailes, C.~Mascolo.
An Ad Hoc Mobility Model Founded on Social Network Theory.
\textit{Proceedings of ACM MSWiM’04}, Venezia, Italy, October 2004.

\begin{notes}
\item Interesting idea, but the actual mobility model is disappointing:
\begin{itemize}
	\item the only thing that's determined according to the social network theory is node groups, and it's not at all clear that the ``social network theory'' is really accurately modeling anything relevant or why the resulting groups would be more realistic than anything else
	\item nodes move within ``the group area'' according to good ol' random way point
	\item groups move in some unspecified manner towards ``randomly chosen goals''
\end{itemize}
\item Evaluation is practically non-existent.
\end{notes}

\bibitem{Tuduce05}
C.~Tuduce, T.~Gross.
A Mobility Model Based on WLAN Traces and its Validation.
\textit{Proceedings of IEEE INFOCOM}, March 2005.

\begin{notes}
\item Mobility model derived from two, two-month campus WLAN traces.
\item Details of the mobility model:
\begin{itemize}
	\item simulation space divided into cells
	\item checkpoint set is set of cells a nodes is going to visit
	\item size of checkpoint set follows a power law
	\item additional cells chosen probabilistically (same, neighbor or non-neighbor)
	\item doesn't consider correlation between checkpoint set size and probability values
	\item later, cells are chosen from checkpoint set uniformly
	\item session durations are independently chosen from a Pareto distribution
	\item inter-session times are strangely chosen from a uniform distribution
\end{itemize}
\item Evaluation:
\begin{itemize}
	\item clever method for choosing the test user set from their data: examine MSE of mobility characteristics (they don't way what) as smaller sets are chosen; choose the smallest set that still will exhibit correct properties.
	\item evaluation of spatial distribution is not so convincing.
\end{itemize}
\end{notes}

\bibitem{Jardosh05}
A.~Jardosh, E.~Belding-Royer, K.~Almeroth, S.~Suri.
Real-World Environment Models for Mobile Network Evaluation.
\textit{IEEE Journal on Selected Areas in Communications}, March 2005.

\begin{notes}
\item Same polygonal obstacle model with Voronoi paths, doorways and destinations~\cite{Jardosh03}.
\item Mobility model additions and variations:
\begin{itemize}
	\item exponential destination selection
	\item note: it would be good to look at how Markov models of mobility from actual traces relate to the exponential destination selection model; are the transition probabilities really exponentially related to distance?
\end{itemize}
\item Signal propagation model:
\begin{itemize}
	\item mixed Friis'/two-ray propagation model
	\item use real observations of signal attenuation through building walls
\end{itemize}
\item Simulation to show differences induced by obstacle model \textit{vs.} other models
\end{notes}

\bibitem{Papadopouli05:duration-association}
M.~Papadopouli, H.~Shen, M.~Spanakis.
Characterizing the duration and association patterns of wireless access in a campus.
\textit{11th European Wireless Conference}, April 2005.

\bibitem{Eagle05}
N.~Eagle, A.~Pentland.
Reality Mining: Sensing Complex Social Systems. 
\textit{Journal of Personal and Ubiquitous Computing}, September 2005 (to appear; see also \URL{http://reality.media.mit.edu/pdfs/realitymining.pdf}).

\bibitem{Jain05}
R.~Jain, A.~Shivaprasad, D.~Lelescu, X.~He.
An empirical model for user registration patterns in a campus wireless LAN.
\textit{Proceedings of ACM MobiCom'05}, Cologne, Germany, September 2005 (to appear; see also \URL{http://delivery.acm.org/10.1145/1060000/1052877/p59-jain.pdf}).

\bibitem{Papadopouli05:client-arrivals}
M.~Papadopouli, H.~Shen, M.~Spanakis.
Modeling client arrivals at access points in wireless campus-wide networks.
Not yet published (see \URL{http://www.csd.uoc.gr/~maria/papers/lanman05-A.pdf}).

\bibitem{Zimmermann05}
H.~Zimmermann , I.~Gruber, C.~Roman. 
A Voronoi-Based Mobility Model For Urban Environments.
Not yet published (see \URL{http://www.lkn.ei.tum.de/forschung/publikationen/dateien/Zimmermann2005AVoronoi-basedMobilityModel.pdf}).

\bibitem{Kim05}
J.~Kim, S.~Bohacek.
A Survey-Based Mobility Model of People for Simulation of Urban Mesh Networks.
Not yet published (see \URL{http://www.eecis.udel.edu/~bohacek/UDelModels/pubs/urbanpedmobility.pdf}).

\end{enumerate}

\end{document}
